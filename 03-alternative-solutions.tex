\section{Альтернативные решения}

\subsection*{Инструменты}

\subsubsection{DeepFaceLab}

Одно из самых популярных решений в области замены лица - DeepFaceLab\cite{deepfacelab}. Оно существует с 2018 года и до сих пор остается передовой и активно развивающейся технологией.

Плюсами являются очень высокое качество наложения и хорошее сохранение похожести лица, большое количество возможностей для подстройки для конкретной пары лиц и видео, высокое разрешение лица.

Несмотря на преимущества, у DFL есть несколько серьезных недостатков, значительно усложняющих его использование. Для работы DFL требует дообучения модели для конкретной пары лиц на достаточно большом наборе данных, включающем изображения со всех ракурсов как для человека на видео, так и для накладываемого персонажа.
Так же дообучение занимает значительное время и требует больших мощностей.
DFL работает по принципу маски, накладывая одно лицо на другое, что делает его очень чувствительным к качеству датасетов. Так же это создает проблему разницы в освещении и цвете между видео и датасетом маски.

Упомянутые выше недостатки проявляются для потенциального пользователя в следующих минусах:
\begin{itemize}
    \item Необходимость в сборе большого датасета для актера и накладываемого персонажа
    \item После сбора датасета необходимо дообучать модель в течение нескольких недель
    \item Датасет должен быть очень качественным - иметь изображения с разнообразных углов и с разным освещением
\end{itemize}

\subsubsection{FSGAN}

\subsection*{Готовые сервисы}

\subsubsection{FaceApp}