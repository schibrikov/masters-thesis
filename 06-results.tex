\section{Результаты}

\subsection{Анализ полученных результатов}

\subsubsection*{Оптимальная нагрузка}

Сервис работает оптимально с точки зрения нагрузки и использования ресурсов, равномерно и максимально эффективно распределяя вычисления между обработчиками.


\subsubsection*{Перекодирование}

Видео перекодируется несколько раз в ходе обработки, что увеличивает потери в качестве, происходит потеря информации из исходного видео. В дальнейшем эти потери хочется минимизировать. Так же это замедляет обработку из-за дополнительных операций. Отказ от множественных перекодирований связан с очень высокой нагрузкой на IO. Огромные объёмы промежуточных данных придётся хранить и передавать по сети, записывать на диски. Это может создать большие ограничения на длину видеофайлов. Идеального решения данной задачи не существует. Каждый из подходов имеет свои минусы.


\subsubsection*{Параллелизм}

Параллельная обработка на нескольких серверах позволяет максимально быстро и оптимально с точки зрения нагрузки выполнить определенный обьем работы. Но с точки зрения множества пользователей это не всегда хорошо. Отработка достаточно длинного файла занимает работой сразу все сервера-обработчики. Если в этот момент на обработку поступает видео другого пользователя, особенно если оно очень короткое, происходит затор. Пользователь с длинным видео готов подождать, но будет сильно задерживать пользователя, например, с картинкой. По сути параллельная обработка заставляет использовать всю пропускную способность сервиса сразу, не оставляя места для параллельной работы с разными клиентами. Так параллельная с точки зрения нагрузки обработка становится последовательной с точки зрения множества пользователей.

Эта проблема решается определенными настройками, позволяющими балансировать между двумя подходами. Например, параллельную обработку для одного пользователя можно ограничить двумя иди тремя потоками. Мы потеряем в максимальной скорости, но при этом оставим место другим обработкам в очереди, даже при обработке очень длинного видеофайла. Алгоритм для работы таких ограничений пока не реализован.

\subsubsection*{Наложение}

Сервис на данный момент имеет проблемы с качеством наложения лиц. Он плохо работает с лицами, находящимися вблизи, т.к. в этом случае сильно увеличивается их площадь и результат сильно теряет в качестве.
Так же в связи с используемым алгоритмом плохо переносится похожесть лица (identity). Наложенное лицо не совсем совпадает с целевым, оно скорее является чем-то средним, полученным из исходного и целевого лица. 