\section{Заключение}

В результате данной работы была разработана система, удовлетворяющая всем требованиям обозначенным при постановке задачи.

Система поддерживает работу множества пользователей, используя веб-интерфейс все пользователи могут одновременно создавать заявки для замены лица, при это заявку могут исполняться несколькими серверами-обработчиками одновременно в порядке очереди.

Разработанный пользовательский интерфейс в достаточной степени интуитивно понятен и позволяет пользоваться сервисом без предварительной подготовки и обучения.

Замена производится лишь по одному изображению исходного лица, загружаемого или сгенерированного, а затем сохраненного в сервисе.

Качество замены лица находятся на достаточно высоком уровне, хотя и имеет некоторые проблемы, описанные более подробно ниже в тексте.

При этом сервисом может воспользоваться любой пользователь, имеющий электронное устройство с доступом в интернет. Никаких особых требований к программному и аппаратному обеспечению пользователя нет.

\subsection{Анализ полученных результатов}

\subsubsection{Оптимальная нагрузка}

В процессе обработки видео задача распределяется по всем доступным серверам-обработчикам, обеспечивая максимально возможную скорость обработки.

Как только освобождается один из обработчиков, он сразу берет на себя следующую задачу для обработки, максимально используя вычислительные мощности для удовлетворения потребностей в обработке.

На данный момент не решена задача приоритезации задач для определенных классов пользователей (например, для крупных клиентов), все задачи приоритезированы одинаково и выполняются максимально быстро.

Проблема такого подхода состоит, например, в том, что платящие клиенты теоретически могут оказаться в очереди за бесплатными клиентами, что не является оптимальным с точки зрения бизнеса.

В планах на будущее присутствует возможность создания нескольких очередей или очереди с приоритетами для решения этой проблемы.

\subsubsection{Перекодирование}

Видео перекодируется несколько раз в ходе обработки, что увеличивает потери в качестве, происходит потеря информации из исходного видео. Так же это замедляет обработку из-за дополнительных операций кодирования/декодирования.

Более оптимальным с точки зрения минимизации потерь качества является передача медиа внутри системы в несжатом виде.
Но этот подход связан с очень высокой нагрузкой на IO (системы ввода-вывода), в частности дисковые хранилища и сеть, а значит заставляет использовать более обьемные и быстрые хранилища для серверов и беспокоится об объеме сетевого трафика внутри кластера.
Например, 10 секунд несжатого 4K (3840х2160) видео с 60 кадрами в секунду обычного качества (SDR с форматом пикселей yuv420p) занимает 7.5 ГБ дискового пространства.
При скорости соединения между узлами в 1 Гбит/с (и соотстветствующей скорости записи/чтения с диска в 125 МБ/с) передача такого файла займет минимум 1 минуту, что при необходимости передач через несколько узлов может занимать значительное время относительно общего времени обработки.

Поэтому данный подход не может быть использован повсеместно и имеет свои ограничения в применимости.

На данный момент мы изучаем возможность применения передачи несжатых медиа.

\subsubsection{Параллелизм}

Параллельная обработка на нескольких серверах позволяет максимально быстро и оптимально с точки зрения нагрузки выполнить определенный обьем работы. Но с точки зрения множества пользователей это не всегда хорошо. Отработка достаточно длинного файла занимает работой сразу все сервера-обработчики. Если в этот момент на обработку поступает видео другого пользователя, особенно если оно очень короткое, происходит затор. Пользователь с длинным видео готов подождать, но будет сильно задерживать пользователя, например, с картинкой. По сути параллельная обработка заставляет использовать всю пропускную способность сервиса сразу, не оставляя места для параллельной работы с разными клиентами. Так параллельная с точки зрения нагрузки обработка становится последовательной с точки зрения множества пользователей.

Эта проблема решается определенными настройками, позволяющими балансировать между двумя подходами. Например, параллельную обработку для одного пользователя можно ограничить двумя иди тремя потоками. Мы потеряем в максимальной скорости, но при этом оставим место другим обработкам в очереди, даже при обработке очень длинного видеофайла. Алгоритм для работы таких ограничений пока не реализован.

\subsubsection{Оценка наложения лиц}

Наша команда постоянно работает над улучшением метрик качества наложения лиц. Тем не менее при нахождении лиц вблизи качество изображения значительно ухудшается в связи с ограниченным разрешением, на котором работает нейронная сеть изменяющая лицо. Алгоритм super-resolution несколько улучшает эту ситуацию, но не исправляет окончательно.

Так же уровень сохранения похожести в разработанной системе недостаточен при использовании сильно различающихся лиц для наложения. Получившееся лицо является неким "средним" между исходным и целевым лицом.