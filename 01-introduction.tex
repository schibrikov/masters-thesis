\section{Введение}

Большая проблема при производстве видеоконтента в рекламе и других областях заключается в привязке к конкретному человеку. Используя конкретного человека, а особенно знаменитость, например, в рекламе, компании имеют дело с постоянными рисками и неудобствами. Это увеличивающиеся расходы (человек становиться более известным и "растет в цене"), сложности в планировании сьемок, опоздания и подобные проблемы.

\subsection*{3D-графика}

Одним из альтернативных вариантов решения являются использование 3D-графики. В этом случае настоящие сьемки практически полностью заменяются на монтаж видео с использованием 3D-моделей и дальнейшей озвучки.
Для этого требуется ручной труд дизайнеров, особое техническое и программное обеспечение.
В целом, данный подход дает огромные возможности, которые активно используются, например в кино, позволяя создавать контент не только с людьми, но и с любыми животными или персонажами, снимать сцены, которые невозможно снять с людьми.
В то же время данный подход имеет множество недостатков, не позволяющих его использовать для производства простого и массового контента. Во-первых - трудоемкость и большие затраты во времени. Процесс создания компьютерной графики может занимать недели и месяцы и требует постоянной работы команды дизайнеров. Минимальные правки, например, одежды, снова требуют очень сложного процесса моделирования в 3D.

\subsection*{Живые сьемки}

Обычным подходом являются полностью живые сьемки с живыми актерами и небольшим монтажем после. Это достаточно дешево, если не привлекать к сьемкам знаменитостей, и не слишком трудоемко. Минусы данного подхода следуют из привязанности к конкретному актеру.
Это постоянно возврастающая стоимость в случае со знаменитостями, человеческий фактор (опоздания, личные проблемы, неподобающий внешний вид), необходимость в макияже, гриме, одежде. Так же живые сьемки не позволяют создавать эффекты, возможные при использовании полноценной 3d-графики.

Несмотря на недостатки, данный подход наиболее широко используется в производстве видеоконтента. Большая часть рекламы, фильмов, видеороликом снято с живыми людьми. 

\subsection*{Автоматическая замена лиц}

С последними исследованиями в области нейронных сетей оказалось, что проблему можно решить с помощью комбинированного подхода. Современные модели нейронных сетей позволяют накладывать лицо на актера на видео автоматически, без ручного труда дизайнеров и специалистов по видеомонтажу.

Такие модели позволяют снимать видео на обычную камеру с любым актером, что гораздо быстрее и дешевле как видеомонтажа с 3d-графикой, так и полноценных сьемок со знаменитостями.

Ранние решения по замене лиц требовали специального обучения модели для конкретной пары актера и лица, а так же обширный набор данных, включающий сьемки головы с разных ракурсов и с разными выражениями лица. Более того, требовалась ручная доработка результата, чтобы считать его удовлетворительным. Самые последние решения дают возможность замены лица по одному снимку и обеспечивают получение результата, практически не требующего доработок.

Хотя решений в этой области достаточно много\cite{DBLP:conf/mm/ChenCNG20}\cite{deepfacelab}\cite{nirkin2019fsgan}, все они имеют свои недостатки. Инструменты, не представленные в виде готового к использованию сервиса, сложны в применении - требуют мощностей и навыков для разворачивания всего необходимого. Существующие сервисы в основном ориентированы на массовых пользователей, а не бизнес, из-за чего имеют недостаточное качество и множество ограничений.

В данной работе рассматривается разработка альтернативного сервиса, использующего новейшие исследования в области замены лиц и предлагающего замену лиц как сервис, ориентированный на бизнес-клиентов.