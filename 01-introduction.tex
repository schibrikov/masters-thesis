\section{Введение}

\subsection{Проблема производства видеоконтента с людьми}

Большая проблема при производстве видеоконтента в рекламе с других областях заключается в привязке к конкретному человеку. Используя конкретного человека, а особенно знаменитость, например, в рекламе, компании имеют дело с постоянными рисками и неудобствами. Это увеличивающиеся расходы (человек становиться более известным и "растет в цене"), сложности в планировании сьемок, опоздания и подобные проблемы.

Этих проблем можно избежать если не привязываться к конкретному физическому человеку. Можно снимать видео с виртуальным персонажем, созданным с помощью компьютерной графики, или купить права на использование изображения человека, и монтировать контент, накладывая лицо на актера. Таким образом можно взять любого актера и сделать его таким, каким привык его видеть зритель.

Нужно понимать, что подобный монтаж с помощью компьютерной графики является очень сложным и дорогим. Хотя это и позволяет избежать многих рисков связанных с человеческим фактором, производство может обходиться не дешевле, чем сьемки знаменитости.

Но с последними исследованиями в области нейронных сетей оказалось, что и эту проблему можно решить. Современные модели позволяют накладывать лицо на актера на видео автоматически, без ручного труда дизайнеров и специалистов по видеомонтажу. Ранние решения для этой цели требовали специального обучения модели для конкретной пары актера и лица, а так же обширный набор данных, включающий сьемки головы с разных ракурсов и с разными выражениями лица. Более того, требовалась ручная доработка результата, чтобы считать его удовлетворительным. Самые последние решения дают возможность замены лица по одному снимку и получение результата, не требующего доработок.