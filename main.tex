\input{preamble}

\title{Разработка распределенной системы обработки видео}
\author{Евгений Щибриков}

\begin{document}

\section{Введение}

Видеоконтент на текущий момент активно применяется во множестве сфер нашей жизни. Большая часть трафика в интернете - это видео. Ежедневно мы смотрим различные фильмы, сериалы, шоу и, что особенно актуально в наше время, ролики в интернете - это могут быть обзоры на товары, клипы, видеоблоги и множество других форматов, представленных на YouTube, в Instagram, TikTok и других площадках. В большей части случаев создатели видеоконтента зарабатывают на рекламе.

Видео отличается от других форматов тем, что предоставляет большую связь с происходящим на нем, так как сочетает как аудио, так и визуальную коммуникацию. Таким образом человек невольно строит более сильные ассоциации с героями роликов или фильмов, а в рекламе конкретный бренд может ассоциироваться с человеком, показываемым на экране.

Но использование людей для съёмок становится большим риском для компаний, постоянно производящих видеоконтент. Например, используя человека как лицо своего бренда компания имеет много рисков: постоянно возрастающая стоимость, опоздания, отказ от сотрудничества и другие возможные проявления человеческого фактора.

Этих проблем при производстве видео-контента можно было бы избежать, если не привязываться к конкретному человеку, а работать с неким виртуальным персонажем. Достичь этого позволяют, например, технологии замены лица.

	

\section{Маркетинговое описание}

Malivar - сервис для замены лица на видео для бизнеса. 
Наш сервис позволяет с помощью одного снимка сделать замену лица на видео. Пользователь выбирает лицо и загружает видео, а на выходе получает видео уже с выбранным лицом.

\section{Техническое описание}

Основа сервиса - ансамбль нейронных сетей, комбинирующий сети для сегментации лица, замены лица (identity injection) и сеть для повышения качества изображения (super resolution).

Очевидно, что такой публичный сервис должен иметь возможность масштабирования в будущем, обработки одновременно большого количества видео для разных клиентов.

Обработка происходит на GPU (видеокарте) и занимает продолжительное время, например обработка 1 минуты видео занимает в среднем около 10 минут.

Исходя из этих ограничений архитектура не может быть монолитной. Сервис состоит из множества компонентов, каждый из которых можно легко масштабировать в соответствии с нагрузкой и ожидаемым количеством пользователей:

\includegraphics[width=\textwidth]{malivar_service_architecture}

\begin{itemize}
	\item Фронтенд - клиентская часть приложения со своим сервером для серверного рендеринга
	\item Бэкенд - основная серверная часть, ответственная за обработку запросов пользователя и соединение остальных компонентов
	\item База данных
	\item Брокер сообщений - для реализации очередей обработки и других асинхронных и параллельных процессов
	\item Сервис(ы) препроцессинга - берут задачу по обработки видео из очереди и осуществляют предварительную и пост-обработку, например перекодирование видео
	\item Сервис(ы) обработки - достают задачи из очереди и проделывают основную работу по обработке видео
\end{itemize}

Так, ключевая ценность проекта содержится в нейронной сети, работающей внутри сервиса обработки. Но для работы сервиса необходима разработка множества других компонентов и вспомогательных сервисов. Я занимался разработкой каждой части системы, кроме самого ядра обработки.




\end{document}