\section{Постановка задачи}

Замена лица - это задача переноса лица с исходного на целевое изображение или видео так, чтобы оно аккуратно заменяло внешность лица на целевом изображении и получался реалистичный результат\cite[стр. 1]{nirkin2019fsgan}.

Главной задачей данной работы является разработка информационной системы, основная функция которой - автоматическая замена лиц на видео. Разработанная ИС должна быть доступна в виде публичного интернет-сервиса, которым может воспользоваться любой человек.
Основной целевой аудиторией сервиса являются бизнес-клиенты, действующие в сферах маркетинга и производства видеоконтента.

Требования к разрабатываемой ИС:

\begin{itemize}
    \item Одновременная работа множества пользователей
    \item Пользовательский интерфейс не требующий отдельного обучения
    \item Замена лица на видео по одному изображению, без необходимости в сборе большого датасета
    \item Хорошее качество замены лица
    \item Отсутствие требований к аппаратному обеспечению пользователя
\end{itemize}


\subsection{Метрики оценки качества замены лица}

Основные метрики оценки результата наложения лиц - качество полученного изображения и уровень похожести лиц.

Под качеством имеются в виду стандартные метрики уровня качества изображения, такие как уровень шума, чистота деталей и потеря деталей в темных областях\cite{burningham2002image}, так и его реалистичность, т.е. уровень способности человеческого глаза отличить получившееся изображение от настоящей фотографии\cite{xue2012understanding}.

Уровень похожести двух лиц мы понимаем интуитивно как совпадение черт лица, позволяющее человеку принять два различных лица за одно и то же\cite{cao2013similarity}\cite{chopra2005learning}\cite{sadovnik2018finding}.