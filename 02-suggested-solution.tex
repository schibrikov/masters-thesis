\section{Предлагаемое решение}

В этой работе рассматривается разработка распределенной многопользовательской системы для наложения лица на видео.

Такая система позволит пользователям за определенную плату получать достаточно качественное наложение лиц без каких-либо требований к навыкам разработки или использования нейронных моделей. 

Процесс наложения лиц достаточно дорогой и требует дорогостоящей инфраструктуры в серверов с графическими процессорами. Поэтому система не может быть бесплатной и должна быть ориентирована на платежеспособных пользователей. Исходя из этого для разработки и продвижения сервиса предлагается стратегия b2b, т.е. позиционирование на бизнес-пользователей, работающих в области создания аудиовизуального контента, например маркетинговые агенства.

\subsection{Альтернативные решения}


\subsubsection*{3D-графика}

Одним из альтернативных вариантов решения являются использование 3d-графики. В этом случае настоящие сьемки практически полностью заменяются на монтаж видео с использованием 3d-моделей и дальнейшей озвучки.
Для этого требуется ручной труд дизайнеров, особое техническое и программное обеспечение.
В целом, данный подход дает огромные возможности, которые активно используются, например в кино, позволяя создавать контент не только с людьми, но и с любыми животными или персонажами, снимать сцены, которые невозможно снять с людьми.
В то же время данный подход имеет множество недостатков, не позволяющих его использовать для производства простого и массового контента. Во-первых - трудоемкость и большие затраты во времени. Процесс создания компьютерной графики может занимать недели и месяцы и требует постоянной работы команды дизайнеров. Минимальные правки, например, одежды, снова требуют очень сложного процесса моделирования в 3D.

\subsubsection*{Живые сьемки}

Обычным подходом являются полностью живые сьемки с живыми актерами и небольшим монтажем после. 